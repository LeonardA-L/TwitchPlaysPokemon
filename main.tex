\documentclass[a4paper]{article}

\usepackage[english]{babel}
\usepackage[utf8x]{inputenc}
\usepackage{amsmath}
\usepackage{graphicx}
\usepackage[colorinlistoftodos]{todonotes}

\title{Organisation in Twitch Plays Pokemon}
\author{Léonard Allain-Launay}

\begin{document}
\maketitle

\begin{abstract}
Your abstract.
\end{abstract}

\section{Introduction}
\textit{Twitch Plays Pokemon} (\textit{TPP}) is a collaborative game that started on the Twitch platform on February 12th 2014. It was also described first by its creator as a \textit{social experiment} to see what would happen if an unlimited number of people were playing the same player in a videogame at the same time. \textit{Twitch} is a web platform allowing videogamers to broadcast their games to the public, to show off their skills, explain strategies, stream e-sports contests, \dots  Each broadcast features a chat where people watching the stream can comment on what happens, and interact with the gamer.

\textit{TPP} hacked this concept by allowing the people watching the stream to interact with the game itself. There was no player, instead everyone was playing. The creator, whose name remain unknown, managed to interpret all messages inserted in the chat of the Twitch stream as Game Boy commands (up, down, left, right, b, a, start, select) and forward them to a Game Boy emulator playing \textit{Pokemon : Red}. This way, what user could enter command and watch in real timethe impact on the game. The problem was, there was one game playing (i.e one controller) and up to more than a hundred thousand users at the same time, controling the same player.

 Chaos followed, lots of wrong moves mare mistakenly made, but ultimately the \textit{TPP} community succeeded in beating the game in sixteen days. This essay is an attempt to list and analyse the factors that made this victory possible, how the community managed to get organised through the chaos, and through problems sometimes unexpected.
\section{Organisation}

\subsection{Communication and context \textit{(or lack thereof)}}

TODO

\subsubsection{The Chat}

TODO

\subsubsection{Too many is just enough}
As the chat, even filtering commands, was flooded with nonsense, the original dedicated way of communicating with everyone was broken. Players then came up with various ways of communicating with each other, quickly setting up :
\begin{itemize}
\item \textbf{A subreddit group :} TODO-Description 
\item \textbf{A live reddit microblog :} TODO-Description 
\item \textbf{An IRC channel :}  TODO-Description 
\item \textbf{A summary on Google doc :}  TODO-Description 
\end{itemize}
These are the main vectors of communication that were available, but possibly hundreds of other arose, as any group of friends playing together might have had their own Facebook group chat and agenda. This could have gone wrong if everyone was suggesting different strategies, but since ultimately the goal was to beat the game, and since every player knew and remembered how to play \textit{Pokemon : Red}, the multiple sources of information were broadcasting an almost unanimous message. Someone who followed the main media had a relatively good vision of the situation of the game.

\subsection{Lack of long term vision}

\subsection{Spambots}

\subsection{Anarchy VS Democracy}

\subsection{Trolls}

\subsection{The Ledge}

Sometimes it is the most unexpected problems that stroke. Usually, a normal person can move around in an orderly fashion. But with hundreds of commands at the same time, the character in \textit{TPP} was caught in a whirlwind. This made a lot of small things become great challenges, such as what became known as \textit{The Ledge}.

TODO-Ledge

As one can imagine, this is not the only ledge in the \textit{Pokemon : Red}.

\section{Conclusion}

\end{document}